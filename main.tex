\documentclass[a4paper,11pt]{article}

\usepackage[activate={true,nocompatibility}]{microtype}
\usepackage{microtype}
\usepackage[T1]{fontenc}
%\usepackage{mlmodern}
%\usepackage{Baskervaldx}
%\usepackage[xcharter]{newtx}
\usepackage{XCharter}
\usepackage{quiver}

\usepackage[backend=bibtex, style=alphabetic]{biblatex}
\addbibresource{main.bib}

\usepackage{amsmath}
\usepackage{amsthm}
\newtheorem{definition}{Definition}[section]
\newtheorem{lemma}{Lemma}[section]
\newtheorem{theorem}{Theorem}[section]

\renewcommand{\baselinestretch}{1.06}
%\usepackage{MnSymbol}
\usepackage{inconsolata}
\usepackage[scaled=0.92]{sourcesanspro}
\usepackage{xcolor}
\usepackage[utf8]{inputenc}
\usepackage[english]{babel}
\usepackage[a4paper, left=1.33in, right=1.33in]{geometry}
\usepackage{calc}
\usepackage{graphicx}

\usepackage{titlesec}
\titleformat{\subsubsection}[runin]
  {\normalfont\itshape}
  {\hspace*{\parindent}\normalfont\bfseries(\thesubsubsection)}
  {1.5\wordsep}
  {}
  [.]

\titlespacing*{\section}{0pt}{2\baselineskip}{1\baselineskip}
\titlespacing*{\subsubsection}
  {0pt}{6pt}{\wordsep}  % left, before, after spacing

\usepackage{mathpartir}
\usepackage{amsmath}

\usepackage{tikz}
\usepackage{wrapfig}

\usepackage[backend=bibtex, style=alphabetic]{biblatex}
%\addbibresource{main.bib}
\renewcommand*{\bibfont}{\small}

\author{Aghilas Y. Boussaa \texttt{<aghilas.boussaa@ens.fr>}}

\title{Étude monadique de l'ordonnancement probabiliste}
\begin{document}
\sloppy
\maketitle
\section{Introduction}
In this section, $\mathcal{C}$ will denote a symmetric tensorial 2-category
whose tensor unit will be noted $1$.

\begin{definition}
  Let $(S,\eta^S, \mu^S)$ and $(T, \eta^T, \mu^T)$ be two monads. Then a
  \emph{distributive law of $S$ over $T$~\cite{beck2006distributive}} is a
  natural transformation $l:TS\Rightarrow ST$ such that the following diagrams
  commute.
% https://q.uiver.app/#q=WzAsMyxbMCwxLCJUUyJdLFsyLDEsIlNUIl0sWzEsMCwiVCJdLFswLDEsImwiLDJdLFsyLDAsIlRcXGV0YV9TIiwyXSxbMiwxLCJcXGV0YV9TVCJdXQ==
\[\begin{tikzcd}
	& T \\
	TS && ST
	\arrow["{T\eta^S}"', from=1-2, to=2-1]
	\arrow["{\eta^ST}", from=1-2, to=2-3]
	\arrow["l"', from=2-1, to=2-3]
\end{tikzcd}\]
% https://q.uiver.app/#q=WzAsMyxbMCwxLCJUUyJdLFsyLDEsIlNUIl0sWzEsMCwiUyJdLFswLDEsImwiLDJdLFsyLDAsIlxcZXRhXlRTIiwyXSxbMiwxLCJTXFxldGFeVCJdXQ==
\[\begin{tikzcd}
	& S \\
	TS && ST
	\arrow["{\eta^TS}"', from=1-2, to=2-1]
	\arrow["{S\eta^T}", from=1-2, to=2-3]
	\arrow["l"', from=2-1, to=2-3]
\end{tikzcd}\]
% https://q.uiver.app/#q=WzAsNSxbMCwwLCJUU1MiXSxbMiwwLCJTVFMiXSxbNCwwLCJTU1QiXSxbMCwyLCJTVCJdLFs0LDIsIlRTIl0sWzAsMSwibFMiXSxbMSwyLCJTbCJdLFswLDMsIlRcXG11XlMiLDJdLFszLDQsImwiLDJdLFsyLDQsIlxcbXVeU1QiXV0=
\[\begin{tikzcd}
	TSS && STS && SST \\
	\\
	ST &&&& TS
	\arrow["lS", from=1-1, to=1-3]
	\arrow["{T\mu^S}"', from=1-1, to=3-1]
	\arrow["Sl", from=1-3, to=1-5]
	\arrow["{\mu^ST}", from=1-5, to=3-5]
	\arrow["l"', from=3-1, to=3-5]
\end{tikzcd}\]

% https://q.uiver.app/#q=WzAsNSxbMCwwLCJUVFMiXSxbMiwwLCJUU1QiXSxbNCwwLCJTVFQiXSxbMCwyLCJTVCJdLFs0LDIsIlRTIl0sWzAsMSwiVGwiXSxbMSwyLCJsVCJdLFswLDMsIlxcbXVeVFMiLDJdLFszLDQsImwiLDJdLFsyLDQsIlNcXG11XlQiXV0=
\[\begin{tikzcd}
	TTS && TST && STT \\
	\\
	ST &&&& TS
	\arrow["Tl", from=1-1, to=1-3]
	\arrow["{\mu^TS}"', from=1-1, to=3-1]
	\arrow["lT", from=1-3, to=1-5]
	\arrow["{S\mu^T}", from=1-5, to=3-5]
	\arrow["l"', from=3-1, to=3-5]
\end{tikzcd}\]
\end{definition}

\begin{theorem}
  Let $(S,\eta^S, \mu^S)$ and $(T, \eta^T, \mu^T)$ be two monads, and let $l$ be
  distributive law of $S$ over $T$. Then the monad $T$ can be lifted to the
  Kleisli category of $S$.
\end{theorem}
\begin{proof}
\end{proof}

We adapt the definitions of \emph{concurrent monoid} from~\cite{rivas2019monads}
to a 2-categorical context. Whiskering forces all monoids in a 2 category to be
ordered.

%% \begin{definition}
%%   An \emph{ordered monoid} is a monoid $(M,\star,e)$ equipped, for all $a,b,c,
%%   d:1\to M$, $\tau: a\Rightarrow b$, and $\lambda: c\Rightarrow d$, with a
%%   2-cell $\omega_{\tau,\lambda}$ such that
%%   % https://q.uiver.app/#q=WzAsNCxbMCwwLCIxIl0sWzIsMCwiTVxcdGltZXMgTSJdLFswLDIsIk1cXHRpbWVzIE0iXSxbMiwyLCJNIl0sWzAsMSwiYlxcdGltZXMgZCJdLFswLDIsImFcXHRpbWVzIGMiLDJdLFsyLDMsIlxcc3RhciIsMl0sWzEsMywiXFxzdGFyIl0sWzIsMSwiXFxvbWVnYV97XFx0YXUsXFxsYW1iZGF9IiwxLHsibGV2ZWwiOjJ9XV0=
%% \[\begin{tikzcd}
%% 	1 && {M\times M} \\
%% 	\\
%% 	{M\times M} && M
%% 	\arrow["{b\times d}", from=1-1, to=1-3]
%% 	\arrow["{a\times c}"', from=1-1, to=3-1]
%% 	\arrow["\star", from=1-3, to=3-3]
%% 	\arrow["{\omega_{\tau,\lambda}}"{description}, Rightarrow, from=3-1, to=1-3]
%% 	\arrow["\star"', from=3-1, to=3-3]
%% \end{tikzcd}\].
%% \end{definition}

\begin{definition}
  A \emph{concurrent monoid} consists of:
  \begin{itemize}
    \item An object $M$ of $\mathcal{C}$;
    \item A morphism $e:1\to M$;
    \item Two morphisms $;$ and $||:M\otimes M\to M$ such that $(M,;,e)$ and
      $(M,||,e)$ are monoids;
    \item A 2-cell $\iota$ such that:
% https://q.uiver.app/#q=WzAsNSxbMCwwLCJNXFx0aW1lcyBNIFxcdGltZXMgTVxcdGltZXMgTSJdLFsyLDAsIk1cXHRpbWVzIE1cXHRpbWVzIE1cXHRpbWVzIE0iXSxbMCwyLCJNXFx0aW1lcyBNIl0sWzQsMCwiTVxcdGltZXMgTSJdLFs0LDIsIk0iXSxbMCwxLCJcXGxhbmcgTSwgdCwgTSBcXHJhbmciXSxbMCwyLCJcXGxhbmd8fCx8fFxccmFuZyIsMl0sWzEsMywiXFxsYW5nIDssO1xccmFuZyJdLFszLDQsInx8Il0sWzIsNCwiOyIsMl0sWzIsMywiXFxpb3RhIiwxLHsibGV2ZWwiOjJ9XV0=
\[\begin{tikzcd}
	{M\otimes M \otimes M\otimes M} && {M\otimes M\otimes M\otimes M} && {M\otimes M} \\
	\\
	{M\otimes M} &&&& M
	\arrow["{M\otimes\sigma\otimes M}", from=1-1, to=1-3]
	\arrow["{||\otimes||}"', from=1-1, to=3-1]
	\arrow["{;\otimes;}", from=1-3, to=1-5]
	\arrow["{||}", from=1-5, to=3-5]
	\arrow["\iota"{description}, Rightarrow, from=3-1, to=1-5]
	\arrow["{;}"', from=3-1, to=3-5]
\end{tikzcd}\]

  \end{itemize}.
\end{definition}

\begin{definition}
  A \emph{bialgebra}~\cite{benson1987shuffle} consists of:
  \begin{itemize}
    \item An object $M$ of $\mathcal{C}$;
    \item Two morphisms $\eta:1\to M$ and $\mu:M\otimes M\to M$ such that $(M,\mu,\eta)$ is a monoid;
    \item Two morphisms $\epsilon:M\to 1$ and $\Delta :M\to M\otimes M$ such that
      $(M,\delta,\epsilon)$ is a comonoid;
  \end{itemize}
  such that the following diagrams commute:
% https://q.uiver.app/#q=WzAsNSxbMCwwLCJNXFx0aW1lcyBNIl0sWzIsMCwiTSJdLFswLDIsIk1cXHRpbWVzIE1cXHRpbWVzIE1cXHRpbWVzIE0iXSxbMCw0LCJNXFx0aW1lcyBNXFx0aW1lcyBNXFx0aW1lcyBNIl0sWzIsNCwiTVxcdGltZXMgTSJdLFsxLDQsIlxcRGVsdGEiXSxbMCwxLCJcXG11Il0sWzAsMiwiXFxsYW5nbGVcXERlbHRhLFxcRGVsdGFcXHJhbmdsZSIsMl0sWzIsMywiXFxsYW5nbGUgTSx0LE1cXHJhbmdsZSIsMl0sWzMsNCwiXFxsYW5nbGVcXG11LFxcbXVcXHJhbmdsZSIsMl1d
\[\begin{tikzcd}
	{M\otimes M} && M \\
	\\
	{M\otimes M\otimes M\otimes M} \\
	\\
	{M\otimes M\otimes M\otimes M} && {M\otimes M}
	\arrow["\mu", from=1-1, to=1-3]
	\arrow["{\Delta\otimes\Delta}"', from=1-1, to=3-1]
	\arrow["\Delta", from=1-3, to=5-3]
	\arrow["{M\otimes\sigma\otimes M}"', from=3-1, to=5-1]
	\arrow["{\mu\otimes\mu}"', from=5-1, to=5-3]
\end{tikzcd}\]
% https://q.uiver.app/#q=WzAsNCxbMCwwLCIxIl0sWzIsMCwiTSJdLFsyLDIsIk1cXHRpbWVzIE0iXSxbMCwyLCIxXFx0aW1lcyAxIl0sWzAsMSwiXFxldGEiXSxbMSwyLCJcXERlbHRhIl0sWzMsMiwiXFxldGFcXHRpbWVzXFxldGEiLDJdLFswLDMsIlxcc2ltIiwyLHsibGV2ZWwiOjIsInN0eWxlIjp7ImhlYWQiOnsibmFtZSI6Im5vbmUifX19XV0=
\[\begin{tikzcd}
	1 && M \\
	\\
	{1\otimes 1} && {M\otimes M}
	\arrow["\eta", from=1-1, to=1-3]
	\arrow["\cong"', equals, from=1-1, to=3-1]
	\arrow["\Delta", from=1-3, to=3-3]
	\arrow["{\eta\otimes\eta}"', from=3-1, to=3-3]
\end{tikzcd}
\begin{tikzcd}
	{M\otimes M} && M \\
	\\
	{1\otimes1} && 1
	\arrow["\mu", from=1-1, to=1-3]
	\arrow["{\epsilon\otimes\epsilon}"', from=1-1, to=3-1]
	\arrow["\epsilon", from=1-3, to=3-3]
	\arrow["\cong"', equals, from=3-1, to=3-3]
\end{tikzcd}
% https://q.uiver.app/#q=WzAsMyxbMCwwLCIxIl0sWzIsMiwiMSJdLFsyLDAsIk0iXSxbMiwxLCJcXGVwc2lsb24iXSxbMCwyLCJcXGV0YSJdLFswLDEsIlxcc2ltIiwyLHsibGV2ZWwiOjIsInN0eWxlIjp7ImhlYWQiOnsibmFtZSI6Im5vbmUifX19XV0=
\begin{tikzcd}
	1 && M \\
	\\
	&& 1
	\arrow["\eta", from=1-1, to=1-3]
	\arrow[equals, from=1-1, to=3-3]
	\arrow["\epsilon", from=1-3, to=3-3]
\end{tikzcd}\]
\end{definition}

\begin{definition}
  A \emph{dagger symmetric monoidal category}~\cite{selinger2007dagger} is a
  category that is both dagger category and a symmetric monoidal category such
  that, for all objects $A,B,C,D$ and all morphisms $f,g$:
  \begin{itemize}
    \item ${(f\otimes g)}^\dagger=f^\dagger\otimes g^\dagger$;
    \item $\alpha_{A,B,C}^\dagger=\alpha_{A,B,C}^{-1}$;
    \item $\lambda_{A}^\dagger=\lambda_{A}^{-1}$;
    \item $\sigma_{A,B}^\dagger=\sigma_{A,B}^{-1}$.
  \end{itemize}
\end{definition}
Let $\text{\textbf{Rel}}$ denote the 2-category of sets, relations and
inclusions with its usual dagger symmetric monoidal structure.

For the sake of readability, we will use juxtaposition for the tensor product.
\begin{lemma}\label{comonmon}
  Let $(M, \star, e)$ be a comonoid in a dagger symmetric monoidal category. Then
  $(M, \star^\dagger, e^\dagger)$ is a monoid.
\end{lemma}
\begin{proof}
  Applying the dagger functor to the diagrams proving that $(M, \star, e)$ is a
  comonoid gives diagrams proving that $(M, \star^\dagger, e^\dagger)$ is a monoid.
\end{proof}

\begin{lemma}\label{epsdageta}
  Let $(M, \mu, \Delta, \eta, \epsilon)$ be a bialgebra in
  $\text{\textbf{Rel}}$. Then $\epsilon^\dagger=\eta$.
\end{lemma}
\begin{proof}
  Applying the dagger functor to the last diagram of the bialgebra definition,
  we have: $\eta^\dagger\circ\epsilon^\dagger = 1$, then
  $\eta=\eta\circ\eta^\dagger\circ\epsilon^\dagger$.
\end{proof}

\begin{theorem}
  Let $(M, \mu, \Delta, \eta, \epsilon)$ be a bialgebra in a dagger symmetric
  monoidal category satisfying the two following diagrams and the equation
  $\epsilon^\dagger=\eta$. Then $(M, \mu, \Delta^\dagger, \eta)$ is a concurrent
  monoid.
  % https://q.uiver.app/#q=WzAsMyxbMCwxLCJNXFxvdGltZXMgTSJdLFs0LDEsIk1cXG90aW1lcyBNIl0sWzIsMCwiTSJdLFswLDEsIiIsMCx7ImxldmVsIjoyLCJzdHlsZSI6eyJoZWFkIjp7Im5hbWUiOiJub25lIn19fV0sWzAsMiwiXFxEZWx0YV5cXGRhZ2dlciIsMl0sWzIsMSwiXFxEZWx0YSIsMl0sWzMsMiwiIiwwLHsic2hvcnRlbiI6eyJzb3VyY2UiOjIwfX1dXQ==
\[\begin{tikzcd}
	&& M \\
	{M\otimes M} &&&& {M\otimes M}
	\arrow["\Delta"', from=1-3, to=2-5]
	\arrow["{\Delta^\dagger}"', from=2-1, to=1-3]
	\arrow[""{name=0, anchor=center, inner sep=0}, equals, from=2-1, to=2-5]
	\arrow[between={0.2}{1}, Rightarrow, from=0, to=1-3]
\end{tikzcd}
  \]
% https://q.uiver.app/#q=WzAsMyxbMCwxLCJNIl0sWzQsMSwiTSJdLFsyLDAsIk1cXG90aW1lcyBNIl0sWzAsMSwiIiwwLHsibGV2ZWwiOjIsInN0eWxlIjp7ImhlYWQiOnsibmFtZSI6Im5vbmUifX19XSxbMCwyLCJcXERlbHRhIiwyXSxbMiwxLCJcXERlbHRhXlxcZGFnZ2VyIiwyXSxbMiwzLCIiLDAseyJzaG9ydGVuIjp7InRhcmdldCI6MjB9fV1d
\[\begin{tikzcd}
	&& {M\otimes M} \\
	M &&&& M
	\arrow["{\Delta^\dagger}"', from=1-3, to=2-5]
	\arrow["\Delta"', from=2-1, to=1-3]
	\arrow[""{name=0, anchor=center, inner sep=0}, equals, from=2-1, to=2-5]
	\arrow[between={0}{0.8}, Rightarrow, from=1-3, to=0]
\end{tikzcd}\]
\end{theorem}
In $\text{\textbf{Rel}}$, the equation is always satisfied by
Lemma~\ref{epsdageta}, the first diagram states that $\Delta^\dagger$ is entire
(each element of $M\otimes M$ is in relation with at lest one element of $M$),
and the second one states that $\Delta^\dagger$ is \emph{functional} (each
element of $M\otimes M$ is in relation with at most one element of $M$). Thus,
in $\text{\textbf{Rel}}$, it suffices that $\Delta^\dagger:M\otimes M\to M$ be a
function.
\begin{proof}
  By Lemma~\ref{comonmon} and the equation in the hypothesis,
  $(M,\Delta^\dagger, \eta)$ is a monoid. Then, by definition of bialgebra
  $(M,\mu,\eta)$ is a monoid. The only thing left to get a concurrent monoid is
  the 2-cell $\iota$.

  By postcomposing the first diagram of the definition of bialgebra with
  $\Delta^\dagger$ and the second hypothesis of the theorem, the following
  diagram commutes.
% https://q.uiver.app/#q=WzAsNyxbMCwwLCJNTSJdLFsyLDAsIk1NTU0iXSxbNCwwLCJNTU1NIl0sWzYsMCwiTU0iXSxbMiwxLCJNIl0sWzQsMiwiTU0iXSxbNiwzLCJNIl0sWzAsMSwiXFxEZWx0YVxcRGVsdGEiXSxbMSwyLCJNXFxzaWdtYSBNIl0sWzIsMywiXFxtdVxcbXUiXSxbMCw0LCJcXG11Il0sWzQsNSwiXFxEZWx0YSJdLFs1LDYsIlxcRGVsdGFeXFxkYWdnZXIiXSxbMyw2LCJcXERlbHRhXlxcZGFnZ2VyIl0sWzAsNiwiXFxtdSIsMix7ImN1cnZlIjo1fV0sWzExLDE0LCIiLDAseyJzaG9ydGVuIjp7InNvdXJjZSI6MjAsInRhcmdldCI6MjB9fV1d
\[\begin{tikzcd}
	MM && MMMM && MMMM && MM \\
	&& M \\
	&&&& MM \\
	&&&&&& M
	\arrow["{\Delta\Delta}", from=1-1, to=1-3]
	\arrow["\mu", from=1-1, to=2-3]
	\arrow[""{name=0, anchor=center, inner sep=0}, "\mu"', curve={height=30pt}, from=1-1, to=4-7]
	\arrow["{M\sigma M}", from=1-3, to=1-5]
	\arrow["{\mu\mu}", from=1-5, to=1-7]
	\arrow["{\Delta^\dagger}", from=1-7, to=4-7]
	\arrow[""{name=1, anchor=center, inner sep=0}, "\Delta", from=2-3, to=3-5]
	\arrow["{\Delta^\dagger}", from=3-5, to=4-7]
	\arrow[between={0.2}{0.8}, Rightarrow, from=1, to=0]
\end{tikzcd}\]

By precomposing the previous diagram by
$\Delta^\dagger\Delta^\dagger\circ M\sigma M$ and the first
hypothesis, the following diagram commutes.
% https://q.uiver.app/#q=WzAsMTMsWzAsMCwiTU1NTSJdLFszLDAsIk1NTU0iXSxbNiwwLCJNTSJdLFsxLDEsIk1NTU0iXSxbMiwyLCJNTSJdLFszLDMsIk1NTU0iXSxbNCw0LCJNTU1NIl0sWzUsNSwiTU0iXSxbNiw2LCJNIl0sWzEsMywiTU1NTSJdLFsyLDQsIk1NTU0iXSxbMyw1LCJNTSJdLFswLDYsIk1NIl0sWzAsMSwiTVxcc2lnbWEgTSJdLFsxLDIsIlxcRGVsdGFeXFxkYWdnZXJcXERlbHRhXlxcZGFnZ2VyIl0sWzAsMywiTVxcc2lnbWEgTSJdLFszLDQsIlxcRGVsdGFeXFxkYWdnZXJcXERlbHRhXlxcZGFnZ2VyIl0sWzQsNSwiXFxEZWx0YVxcRGVsdGEiXSxbNSw2LCJNXFxzaWdtYSBNIl0sWzYsNywiXFxtdVxcbXUiXSxbNyw4LCJcXERlbHRhXlxcZGFnZ2VyIl0sWzIsOCwiXFxtdSJdLFs1LDIsIiIsMSx7InNob3J0ZW4iOnsic291cmNlIjoxMCwidGFyZ2V0IjoxMH0sImxldmVsIjoyfV0sWzAsOSwiTVxcc2lnbWEgTSIsMl0sWzksMTAsIk1cXHNpZ21hIE0iLDJdLFsxMCwxMSwiXFxtdVxcbXUiLDJdLFsxMSw4LCJcXERlbHRhXlxcZGFnZ2VyIl0sWzAsMTIsIlxcbXVcXG11IiwyXSxbMTIsOCwiXFxEZWx0YV5cXGRhZ2dlciIsMl0sWzEwLDUsIiIsMix7InNob3J0ZW4iOnsic291cmNlIjoxMCwidGFyZ2V0IjoxMH0sImxldmVsIjoyfV1d
\[\begin{tikzcd}[sep=tiny]
	MMMM &&& MMMM &&& MM \\
	& MMMM \\
	&& MM \\
	& MMMM && MMMM \\
	&& MMMM && MMMM \\
	&&& MM && MM \\
	MM &&&&&& M
	\arrow["{M\sigma M}", from=1-1, to=1-4]
	\arrow["{M\sigma M}", from=1-1, to=2-2]
	\arrow["{M\sigma M}"', from=1-1, to=4-2]
	\arrow["{\mu\mu}"', from=1-1, to=7-1]
	\arrow["{\Delta^\dagger\Delta^\dagger}", from=1-4, to=1-7]
	\arrow["\mu", from=1-7, to=7-7]
	\arrow["{\Delta^\dagger\Delta^\dagger}", from=2-2, to=3-3]
	\arrow["{\Delta\Delta}", from=3-3, to=4-4]
	\arrow["{M\sigma M}"', from=4-2, to=5-3]
	\arrow[between={0.1}{0.9}, Rightarrow, from=4-4, to=1-7]
	\arrow["{M\sigma M}", from=4-4, to=5-5]
	\arrow[between={0.1}{0.9}, Rightarrow, from=5-3, to=4-4]
	\arrow["{\mu\mu}"', from=5-3, to=6-4]
	\arrow["{\mu\mu}", from=5-5, to=6-6]
	\arrow["{\Delta^\dagger}", from=6-4, to=7-7]
	\arrow["{\Delta^\dagger}", from=6-6, to=7-7]
	\arrow["{\Delta^\dagger}"', from=7-1, to=7-7]
\end{tikzcd}\]

This diagram gives, by composition, a 2-cell making $(M, \mu, \Delta^\dagger,
\eta)$ a concurrent monoid.
\end{proof}

%% \begin{lemma}
%%   All monoids in $Rel$ are ordered.
%% \end{lemma}
%% \begin{proof}
%%   Let $(M, \star, e)$ be a monoid in $Rel$.
%% \end{proof}
\printbibliography
\end{document}
