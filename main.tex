\documentclass[a4paper,11pt]{article}

\usepackage[activate={true,nocompatibility}]{microtype}
\usepackage{microtype}
\usepackage[T1]{fontenc}
%\usepackage{mlmodern}
%\usepackage{Baskervaldx}
%\usepackage[xcharter]{newtx}
\usepackage{XCharter}
\usepackage{quiver}
\usepackage{stackengine}

\usepackage[backend=bibtex, style=alphabetic]{biblatex}
\addbibresource{main.bib}

\usepackage{amsmath}
\usepackage{amsthm}
\usepackage{amssymb}
\newtheorem{definition}{Definition}[section]
\newtheorem{lemma}{Lemma}[section]
\newtheorem{theorem}{Theorem}[section]
\theoremstyle{remark}
\newtheorem*{example}{Example}

\renewcommand{\baselinestretch}{1.06}
%\usepackage{MnSymbol}
\usepackage{inconsolata}
\usepackage[scaled=0.92]{sourcesanspro}
\usepackage{xcolor}
\usepackage[utf8]{inputenc}
\usepackage[english]{babel}
\usepackage[a4paper, left=1.33in, right=1.33in]{geometry}
\usepackage{calc}
\usepackage{graphicx}

\usepackage{titlesec}
\titleformat{\subsubsection}[runin]
  {\normalfont\itshape}
  {\hspace*{\parindent}\normalfont\bfseries(\thesubsubsection)}
  {1.5\wordsep}
  {}
  [.]

\titlespacing*{\section}{0pt}{2\baselineskip}{1\baselineskip}
\titlespacing*{\subsubsection}
  {0pt}{6pt}{\wordsep}  % left, before, after spacing

\usepackage{mathpartir}
\usepackage{amsmath}

\usepackage{tikz}
\usepackage{wrapfig}

\usepackage[backend=bibtex, style=alphabetic]{biblatex}
%\addbibresource{main.bib}
\renewcommand*{\bibfont}{\small}
\newcommand{\rightarrowbar}{\mathrel{+\!\!\!\!\!\!\rightarrow}}
\author{Aghilas Y. Boussaa \texttt{<aghilas.boussaa@ens.fr>}}

\title{Étude monadique de l'ordonnancement probabiliste}
\begin{document}
\sloppy
\maketitle
\section{Introduction}
In this section, $\mathcal{C}$ will denote a symmetric tensorial 2-category
whose tensor unit will be noted $1$.

\begin{definition}
  A \emph{monad} consists of a functor $T$ equipped with two natural
  transformations $\eta:\text{id}\Rightarrow T$ (called \emph{unit}) and
  $\mu:T^2\Rightarrow T$ (called \emph{multiplication}) such that the following
  diagrams commute.
% https://q.uiver.app/#q=WzAsNCxbMCwwLCJUIl0sWzEsMCwiVFQiXSxbMiwwLCJUIl0sWzEsMSwiVCJdLFswLDEsIlxcZXRhIFQiXSxbMiwxLCJUXFxldGEiLDJdLFsxLDMsIlxcbXUiLDJdLFszLDIsIiIsMix7ImxldmVsIjoyLCJzdHlsZSI6eyJoZWFkIjp7Im5hbWUiOiJub25lIn19fV0sWzAsMywiIiwyLHsibGV2ZWwiOjIsInN0eWxlIjp7ImhlYWQiOnsibmFtZSI6Im5vbmUifX19XV0=
\[\begin{tikzcd}
	T & TT & T \\
	& T
	\arrow["{\eta T}", from=1-1, to=1-2]
	\arrow[equals, from=1-1, to=2-2]
	\arrow["\mu"', from=1-2, to=2-2]
	\arrow["{T\eta}"', from=1-3, to=1-2]
	\arrow[equals, from=2-2, to=1-3]
\end{tikzcd}
% https://q.uiver.app/#q=WzAsNCxbMCwwLCJUVFQiXSxbMSwwLCJUVCJdLFswLDEsIlRUIl0sWzEsMSwiVCJdLFswLDEsIlRcXG11Il0sWzAsMiwiXFxtdSBUIiwyXSxbMiwzLCJcXG11IiwyXSxbMSwzLCJcXG11Il1d
\begin{tikzcd}
	TTT & TT \\
	TT & T
	\arrow["{T\mu}", from=1-1, to=1-2]
	\arrow["{\mu T}"', from=1-1, to=2-1]
	\arrow["\mu", from=1-2, to=2-2]
	\arrow["\mu"', from=2-1, to=2-2]
\end{tikzcd}\]
\end{definition}
\begin{example}
  The power set functor $\mathcal{P}$ (mapping each set to each power set and
  each function to the associated direct image function) is a monad with:
  \begin{itemize}
    \item $\eta_X(x)=\{x\}$;
    \item $\mu_X(A)=\bigcup_{a\in A} a$.
  \end{itemize}
  Let $\Sigma$ be a finite set. The writer monad $-\times\Sigma^*$ on the
  alphabet $\Sigma$ is defined by:
  \begin{itemize}
    \item $\left(f\times\Sigma^*\right)((a,w)) = (f(a),w)$ for $f:A\to B$;
    \item $\eta_X(x)=(x,\epsilon)$;
    \item $\mu_X((a,w)w')=(a, ww')$.
  \end{itemize}
\end{example}

\begin{definition}
  Let $(S,\eta^S, \mu^S)$ and $(T, \eta^T, \mu^T)$ be two monads. Then a
  \emph{distributive law of $S$ over $T$~\cite{beck2006distributive}} is a
  natural transformation $l:TS\Rightarrow ST$ such that the following diagrams
  commute.
% https://q.uiver.app/#q=WzAsMyxbMCwxLCJUUyJdLFsyLDEsIlNUIl0sWzEsMCwiVCJdLFswLDEsImwiLDJdLFsyLDAsIlRcXGV0YV9TIiwyXSxbMiwxLCJcXGV0YV9TVCJdXQ==
\[\begin{tikzcd}
	& T \\
	TS && ST
	\arrow["{T\eta^S}"', from=1-2, to=2-1]
	\arrow["{\eta^ST}", from=1-2, to=2-3]
	\arrow["l"', from=2-1, to=2-3]
\end{tikzcd}
% https://q.uiver.app/#q=WzAsMyxbMCwxLCJUUyJdLFsyLDEsIlNUIl0sWzEsMCwiUyJdLFswLDEsImwiLDJdLFsyLDAsIlxcZXRhXlRTIiwyXSxbMiwxLCJTXFxldGFeVCJdXQ==
\begin{tikzcd}
	& S \\
	TS && ST
	\arrow["{\eta^TS}"', from=1-2, to=2-1]
	\arrow["{S\eta^T}", from=1-2, to=2-3]
	\arrow["l"', from=2-1, to=2-3]
\end{tikzcd}\]
% https://q.uiver.app/#q=WzAsNSxbMCwwLCJUU1MiXSxbMiwwLCJTVFMiXSxbNCwwLCJTU1QiXSxbMCwyLCJTVCJdLFs0LDIsIlRTIl0sWzAsMSwibFMiXSxbMSwyLCJTbCJdLFswLDMsIlRcXG11XlMiLDJdLFszLDQsImwiLDJdLFsyLDQsIlxcbXVeU1QiXV0=
\[\begin{tikzcd}
	TSS && STS && SST \\
	\\
	TS &&&& ST
	\arrow["lS", from=1-1, to=1-3]
	\arrow["{T\mu^S}"', from=1-1, to=3-1]
	\arrow["Sl", from=1-3, to=1-5]
	\arrow["{\mu^ST}", from=1-5, to=3-5]
	\arrow["l"', from=3-1, to=3-5]
\end{tikzcd}\]

% https://q.uiver.app/#q=WzAsNSxbMCwwLCJUVFMiXSxbMiwwLCJUU1QiXSxbNCwwLCJTVFQiXSxbMCwyLCJTVCJdLFs0LDIsIlRTIl0sWzAsMSwiVGwiXSxbMSwyLCJsVCJdLFswLDMsIlxcbXVeVFMiLDJdLFszLDQsImwiLDJdLFsyLDQsIlNcXG11XlQiXV0=
\[\begin{tikzcd}
	TTS && TST && STT \\
	\\
	TS &&&& ST
	\arrow["Tl", from=1-1, to=1-3]
	\arrow["{\mu^TS}"', from=1-1, to=3-1]
	\arrow["lT", from=1-3, to=1-5]
	\arrow["{S\mu^T}", from=1-5, to=3-5]
	\arrow["l"', from=3-1, to=3-5]
\end{tikzcd}\]
\end{definition}

On the category $\text{\textbf{Set}}$, let $S$ be the power set monad, and let
$T$ be the writer monad associated to a set $\Sigma$. Then, the following $l$ is
a distributive law of $S$ over $T$:
\[
l_X((A\subseteq X, w\in\Sigma^\star)) = \left\{(a,w) | a\in A\right\}.
\]

\begin{theorem}
  Let $(S,\eta^S, \mu^S)$ and $(T, \eta^T, \mu^T)$ be two monads, and let $l$ be
  distributive law of $S$ over $T$. Then the monad $T$ can be lifted to
  $\mathcal{C}_S$, the Kleisli category of $S$.
\end{theorem}
\begin{proof}
  Define $\tilde{T}$ by:
  \begin{itemize}
    \item $\tilde{T}(A:\mathcal{C}_S)=TA$;
    \item $\tilde{T}(f:A\to SB)=l_B\circ Tf$.
  \end{itemize}

  Let us show that $\tilde{T}$ is a functor. Let $A$ be an object of
  $\mathcal{C}$, then $\tilde{T}\eta^S_A=l_{TA}\circ T\eta^S_A=\eta^S_{TA}$ as
  $l$ is a distributive law: $\tilde{T}$ preserves identities. Let $f:A\to SB$
  and $g:B\to SC$, then the follwoing diagram commutes: by naturality of $l$ for
  the left square and by definition of distributive law for the right rectangle.
  % https://q.uiver.app/#q=WzAsOCxbMCwwLCJUQSJdLFsxLDAsIlRTQiJdLFsyLDAsIlRTU0MiXSxbMSwxLCJTVEIiXSxbMiwxLCJTVFNDIl0sWzMsMSwiU1NUQyJdLFs0LDEsIlNUQyJdLFs0LDAsIlRTQyJdLFswLDEsIlRmIl0sWzEsMiwiVFNnIl0sWzMsNCwiU1RnIl0sWzQsNSwiU2xfQyJdLFs1LDYsIlxcbXVeU197VEN9Il0sWzEsMywibF9CIl0sWzIsNCwibF97U0N9Il0sWzIsNywiVFxcbXVfQ157U30iXSxbNyw2LCJsX0MiXV0=
\[\begin{tikzcd}
	TA & TSB & TSSC && TSC \\
	& STB & STSC & SSTC & STC
	\arrow["Tf", from=1-1, to=1-2]
	\arrow["TSg", from=1-2, to=1-3]
	\arrow["{l_B}", from=1-2, to=2-2]
	\arrow["{T\mu_C^{S}}", from=1-3, to=1-5]
	\arrow["{l_{SC}}", from=1-3, to=2-3]
	\arrow["{l_C}", from=1-5, to=2-5]
	\arrow["STg", from=2-2, to=2-3]
	\arrow["{Sl_C}", from=2-3, to=2-4]
	\arrow["{\mu^S_{TC}}", from=2-4, to=2-5]
\end{tikzcd}\]
Thus, $\tilde{T}$ preserves composition.

Let us show that $\eta^{\tilde{T}}=(\eta^S T)\eta^T:\text{id}\Rightarrow
\tilde{T}$ and $\mu^{\tilde{T}}=(\eta^S T):\mu^T$ endow $\tilde{T}$ with a monad
structure. The two following diagrams in $\mathcal{C}$ commute for all object
$X:\mathcal{C}$ because of the monad structure of $S$ and $T$ and the
distributivity diagrams, and they correspond in $\mathcal{C}_S$ to the diagrams
defining monads.

%%%%% https://q.uiver.app/#q=WzAsMTEsWzAsMSwiVCJdLFsxLDEsIlRUWCJdLFszLDEsIlNUVCJdLFs0LDAsIlRTVCJdLFs1LDEsIlRUIl0sWzYsMSwiVCJdLFszLDIsIlNUIl0sWzMsMywiU1NUIl0sWzMsNCwiU1QiXSxbMiwyLCJTVCJdLFs0LDIsIlNUIl0sWzAsMSwiXFxldGFeVFQiXSxbMSwyLCJcXGV0YV5TVFQiXSxbMywyLCJsVCIsMl0sWzQsMywiVFxcZXRhXlNUIiwyXSxbNSw0LCJUXFxldGFeVCIsMl0sWzQsMiwiXFxldGFeU1RUIiwyXSxbMiw2LCJTXFxtdV5UIiwxXSxbNiw3LCJTXFxldGFeU1QiXSxbNyw4LCJcXG11XlNUIl0sWzAsOCwiXFxldGFeU1QiLDJdLFs1LDgsIlxcZXRhXlNUIl0sWzYsOCwiIiwwLHsiY3VydmUiOjIsImxldmVsIjoyLCJzdHlsZSI6eyJoZWFkIjp7Im5hbWUiOiJub25lIn19fV0sWzAsOSwiXFxldGFeU1QiXSxbOSwyLCJTXFxldGFeVFQiXSxbOSw2LCIiLDAseyJsZXZlbCI6Miwic3R5bGUiOnsiaGVhZCI6eyJuYW1lIjoibm9uZSJ9fX1dLFs2LDEwLCIiLDAseyJsZXZlbCI6Miwic3R5bGUiOnsiaGVhZCI6eyJuYW1lIjoibm9uZSJ9fX1dLFs1LDEwLCJcXGV0YV5TVCIsMl0sWzEwLDIsIlNUXFxldGFeVCIsMl1d

% https://q.uiver.app/#q=WzAsMTEsWzAsMSwiVFgiXSxbMSwxLCJUVFgiXSxbMywxLCJTVFRYIl0sWzQsMCwiVFNUWCJdLFs1LDEsIlRUWCJdLFs2LDEsIlRYIl0sWzMsMiwiU1RYIl0sWzMsMywiU1NUWCJdLFszLDQsIlNUWCJdLFsyLDIsIlNUWCJdLFs0LDIsIlNUWCJdLFswLDEsIlxcZXRhXlRfe1RYfSJdLFsxLDIsIlxcZXRhX3tUVFh9XlMiXSxbMywyLCJsX3tUWH0iLDJdLFs0LDMsIlRcXGV0YV97VFh9XlMiLDJdLFs1LDQsIlRcXGV0YV5UX1giLDJdLFs0LDIsIlxcZXRhX3tUVFh9XlMiLDJdLFsyLDYsIlNcXG11X1heVCIsMV0sWzYsNywiU1xcZXRhXlNfe1RYfSJdLFs3LDgsIlxcbXVeU197VFh9Il0sWzAsOCwiXFxldGFeU197VFh9IiwyXSxbNSw4LCJcXGV0YV97VFh9XlMiXSxbNiw4LCIiLDAseyJjdXJ2ZSI6MiwibGV2ZWwiOjIsInN0eWxlIjp7ImhlYWQiOnsibmFtZSI6Im5vbmUifX19XSxbMCw5LCJcXGV0YV5TX3tUWH0iXSxbOSwyLCJTXFxldGFeVF97VFh9Il0sWzksNiwiIiwwLHsibGV2ZWwiOjIsInN0eWxlIjp7ImhlYWQiOnsibmFtZSI6Im5vbmUifX19XSxbNiwxMCwiIiwwLHsibGV2ZWwiOjIsInN0eWxlIjp7ImhlYWQiOnsibmFtZSI6Im5vbmUifX19XSxbNSwxMCwiXFxldGFeU197VFh9IiwyXSxbMTAsMiwiU1RcXGV0YV5UX1giLDJdXQ==
\[\begin{tikzcd}
	&&&& TSTX \\
	TX & TTX && STTX && TTX & TX \\
	&& STX & STX & STX \\
	&&& SSTX \\
	&&& STX
	\arrow["{l_{TX}}"', from=1-5, to=2-4]
	\arrow["{\eta^T_{TX}}", from=2-1, to=2-2]
	\arrow["{\eta^S_{TX}}", from=2-1, to=3-3]
	\arrow["{\eta^S_{TX}}"', from=2-1, to=5-4]
	\arrow["{\eta_{TTX}^S}", from=2-2, to=2-4]
	\arrow["{S\mu_X^T}"{description}, from=2-4, to=3-4]
	\arrow["{T\eta_{TX}^S}"', from=2-6, to=1-5]
	\arrow["{\eta_{TTX}^S}"', from=2-6, to=2-4]
	\arrow["{T\eta^T_X}"', from=2-7, to=2-6]
	\arrow["{\eta^S_{TX}}"', from=2-7, to=3-5]
	\arrow["{\eta_{TX}^S}", from=2-7, to=5-4]
	\arrow["{S\eta^T_{TX}}", from=3-3, to=2-4]
	\arrow[equals, from=3-3, to=3-4]
	\arrow[equals, from=3-4, to=3-5]
	\arrow["{S\eta^S_{TX}}", from=3-4, to=4-4]
	\arrow[curve={height=24pt}, equals, from=3-4, to=5-4]
	\arrow["{ST\eta^T_X}"', from=3-5, to=2-4]
	\arrow["{\mu^S_{TX}}", from=4-4, to=5-4]
\end{tikzcd}\]

%%%%% https://q.uiver.app/#q=WzAsMTMsWzAsMSwiVFRUIl0sWzEsMSwiVFQiXSxbMiwwLCJUU1QiXSxbMywxLCJTVFQiXSxbMyw0LCJTVCJdLFswLDIsIlRUIl0sWzAsNCwiU1RUIl0sWzEsNCwiU1QiXSxbMiw0LCJTU1QiXSxbMywyLCJTVCJdLFszLDMsIlNTVCJdLFsyLDMsIlNUIl0sWzEsMiwiVCJdLFswLDEsIlRcXG11XlQiXSxbMSwyLCJUXFxldGFeU1QiXSxbMCw1LCJcXG11XlRfe1RYfSIsMl0sWzUsNiwiXFxldGFeU1RUIiwyXSxbNiw3LCJTXFxtdV5UIiwyXSxbNyw4LCJTXFxldGFeU1QiXSxbOCw0LCJcXG11XlNUIl0sWzIsMywibFQiXSxbMSwzLCJcXGV0YV5TVFQiLDFdLFszLDksIlNcXG11XlQiXSxbOSwxMCwiU1xcZXRhXlNUIiwxXSxbMTAsNCwiXFxtdV5TVCIsMl0sWzksNCwiIiwyLHsiY3VydmUiOi0zLCJsZXZlbCI6Miwic3R5bGUiOnsiaGVhZCI6eyJuYW1lIjoibm9uZSJ9fX1dLFs3LDQsIiIsMix7ImN1cnZlIjozLCJsZXZlbCI6Miwic3R5bGUiOnsiaGVhZCI6eyJuYW1lIjoibm9uZSJ9fX1dLFs3LDExLCIiLDEseyJsZXZlbCI6Miwic3R5bGUiOnsiaGVhZCI6eyJuYW1lIjoibm9uZSJ9fX1dLFsxMSw5LCIiLDEseyJsZXZlbCI6Miwic3R5bGUiOnsiaGVhZCI6eyJuYW1lIjoibm9uZSJ9fX1dLFsxMiwxMSwiXFxldGFeU1QiXSxbMywxMSwiU1xcbXVeVCIsMV0sWzEsMTIsIlxcbXVeVCJdLFs1LDEyLCJcXG11XlQiLDJdLFs2LDExLCJTXFxtdV5UIiwxXV0=

% https://q.uiver.app/#q=WzAsMTMsWzAsMSwiVFRUWCJdLFsxLDEsIlRUWCJdLFsyLDAsIlRTVFgiXSxbMywxLCJTVFRYIl0sWzMsNCwiU1RYIl0sWzAsMiwiVFRYIl0sWzAsNCwiU1RUWCJdLFsxLDQsIlNUWCJdLFsyLDQsIlNTVFgiXSxbMywyLCJTVFgiXSxbMywzLCJTU1RYIl0sWzIsMywiU1RYIl0sWzEsMiwiVFgiXSxbMCwxLCJUXFxtdV5UX1giXSxbMSwyLCJUXFxldGFeU197VFh9Il0sWzAsNSwiXFxtdV5UX3tUWH0iLDJdLFs1LDYsIlxcZXRhX3tUVFh9XlMiLDJdLFs2LDcsIlNcXG11XlRfWCIsMl0sWzcsOCwiU1xcZXRhXlNfe1RYfSJdLFs4LDQsIlxcbXVeU197VFh9Il0sWzIsMywibF97VFh9Il0sWzEsMywiXFxldGFfe1RUWH1eUyIsMV0sWzMsOSwiU1xcbXVfWF5UIl0sWzksMTAsIlNcXGV0YV97VFh9XlMiLDJdLFsxMCw0LCJcXG11XlNfe1RYfSIsMl0sWzksNCwiIiwyLHsiY3VydmUiOi0zLCJsZXZlbCI6Miwic3R5bGUiOnsiaGVhZCI6eyJuYW1lIjoibm9uZSJ9fX1dLFs3LDQsIiIsMix7ImN1cnZlIjozLCJsZXZlbCI6Miwic3R5bGUiOnsiaGVhZCI6eyJuYW1lIjoibm9uZSJ9fX1dLFs3LDExLCIiLDEseyJsZXZlbCI6Miwic3R5bGUiOnsiaGVhZCI6eyJuYW1lIjoibm9uZSJ9fX1dLFsxMSw5LCIiLDEseyJsZXZlbCI6Miwic3R5bGUiOnsiaGVhZCI6eyJuYW1lIjoibm9uZSJ9fX1dLFsxMiwxMSwiXFxldGFfe1RYfV5TIl0sWzMsMTEsIlNcXG11XlRfWCIsMV0sWzEsMTIsIlxcbXVeVF9YIl0sWzUsMTIsIlxcbXVfWF5UIiwyXSxbNiwxMSwiU1xcbXVeVF9YIiwxXV0=
\[\begin{tikzcd}
	&& TSTX \\
	TTTX & TTX && STTX \\
	TTX & TX && STX \\
	&& STX & SSTX \\
	STTX & STX & SSTX & STX
	\arrow["{l_{TX}}", from=1-3, to=2-4]
	\arrow["{T\mu^T_X}", from=2-1, to=2-2]
	\arrow["{\mu^T_{TX}}"', from=2-1, to=3-1]
	\arrow["{T\eta^S_{TX}}", from=2-2, to=1-3]
	\arrow["{\eta_{TTX}^S}"{description}, from=2-2, to=2-4]
	\arrow["{\mu^T_X}", from=2-2, to=3-2]
	\arrow["{S\mu_X^T}", from=2-4, to=3-4]
	\arrow["{S\mu^T_X}"{description}, from=2-4, to=4-3]
	\arrow["{\mu_X^T}"', from=3-1, to=3-2]
	\arrow["{\eta_{TTX}^S}"', from=3-1, to=5-1]
	\arrow["{\eta_{TX}^S}", from=3-2, to=4-3]
	\arrow["{S\eta_{TX}^S}"', from=3-4, to=4-4]
	\arrow[curve={height=-18pt}, equals, from=3-4, to=5-4]
	\arrow[equals, from=4-3, to=3-4]
	\arrow["{\mu^S_{TX}}"', from=4-4, to=5-4]
	\arrow["{S\mu^T_X}"{description}, from=5-1, to=4-3]
	\arrow["{S\mu^T_X}"', from=5-1, to=5-2]
	\arrow[equals, from=5-2, to=4-3]
	\arrow["{S\eta^S_{TX}}", from=5-2, to=5-3]
	\arrow[curve={height=18pt}, equals, from=5-2, to=5-4]
	\arrow["{\mu^S_{TX}}", from=5-3, to=5-4]
\end{tikzcd}\]
\end{proof}

This theorem allows us to lift the writer monad to the Kleisli category of the
power set monad: $\textbf{Rel}$.

\begin{definition}
  $\dashrightarrow$
  A \emph{monoid} in a category $\mathcal{C}$ consists of:
  \begin{itemize}
    \item An object $M$ of $\mathcal{C}$;
    \item A morphism $e:1\to M$;
    \item A morphism $\star:M\otimes M\to M$;
  \end{itemize}
  such that the following diagrams commute:
% https://q.uiver.app/#q=WzAsNSxbMCwwLCIoTVxcb3RpbWVzIE0pXFxvdGltZXMgTSJdLFsyLDAsIk1cXG90aW1lcyAoTVxcb3RpbWVzIE0pIl0sWzQsMCwiTVxcb3RpbWVzIE0iXSxbMSwxLCJNXFxvdGltZXMgTSJdLFszLDEsIk0iXSxbMCwxLCJcXGFscGhhIl0sWzEsMiwiTVxcb3RpbWVzXFxtdSJdLFswLDMsIlxcbXVcXG90aW1lcyBNIiwyXSxbMiw0LCJcXG11Il0sWzMsNCwiXFxtdSIsMl1d
\[\begin{tikzcd}[sep=tiny]
	{(M\otimes M)\otimes M} && {M\otimes (M\otimes M)} && {M\otimes M} \\
	& {M\otimes M} && M
	\arrow["\alpha", from=1-1, to=1-3]
	\arrow["{\mu\otimes M}"', from=1-1, to=2-2]
	\arrow["{M\otimes\mu}", from=1-3, to=1-5]
	\arrow["\mu", from=1-5, to=2-4]
	\arrow["\mu"', from=2-2, to=2-4]
\end{tikzcd}\]
% https://q.uiver.app/#q=WzAsNCxbMCwwLCIxXFxvdGltZXMgTSJdLFsxLDAsIk1cXG90aW1lcyBNIl0sWzIsMCwiTVxcb3RpbWVzIDEiXSxbMSwxLCJNIl0sWzAsMSwiZVxcb3RpbWVzIE0iXSxbMiwxLCJNXFxvdGltZXMgZSIsMl0sWzAsMywiXFxsYW1iZGEiLDJdLFsxLDMsIlxcc3RhciIsMl0sWzIsMywiXFxyaG8iXV0=
\[\begin{tikzcd}
	{1\otimes M} & {M\otimes M} & {M\otimes 1} \\
	& M
	\arrow["{e\otimes M}", from=1-1, to=1-2]
	\arrow["\lambda"', from=1-1, to=2-2]
	\arrow["\star"', from=1-2, to=2-2]
	\arrow["{M\otimes e}"', from=1-3, to=1-2]
	\arrow["\rho", from=1-3, to=2-2]
\end{tikzcd}\]
\end{definition}

We adapt the definitions of \emph{concurrent monoid} from~\cite{rivas2019monads}
to a 2-categorical context. Whiskering forces all monoids in a 2 category to be
ordered.

%% \begin{definition}
%%   An \emph{ordered monoid} is a monoid $(M,\star,e)$ equipped, for all $a,b,c,
%%   d:1\to M$, $\tau: a\Rightarrow b$, and $\lambda: c\Rightarrow d$, with a
%%   2-cell $\omega_{\tau,\lambda}$ such that
%%   % https://q.uiver.app/#q=WzAsNCxbMCwwLCIxIl0sWzIsMCwiTVxcdGltZXMgTSJdLFswLDIsIk1cXHRpbWVzIE0iXSxbMiwyLCJNIl0sWzAsMSwiYlxcdGltZXMgZCJdLFswLDIsImFcXHRpbWVzIGMiLDJdLFsyLDMsIlxcc3RhciIsMl0sWzEsMywiXFxzdGFyIl0sWzIsMSwiXFxvbWVnYV97XFx0YXUsXFxsYW1iZGF9IiwxLHsibGV2ZWwiOjJ9XV0=
%% \[\begin{tikzcd}
%% 	1 && {M\times M} \\
%% 	\\
%% 	{M\times M} && M
%% 	\arrow["{b\times d}", from=1-1, to=1-3]
%% 	\arrow["{a\times c}"', from=1-1, to=3-1]
%% 	\arrow["\star", from=1-3, to=3-3]
%% 	\arrow["{\omega_{\tau,\lambda}}"{description}, Rightarrow, from=3-1, to=1-3]
%% 	\arrow["\star"', from=3-1, to=3-3]
%% \end{tikzcd}\].
%% \end{definition}

\begin{definition}
  A \emph{concurrent monoid} consists of:
  \begin{itemize}
    \item An object $M$ of $\mathcal{C}$;
    \item A morphism $e:1\to M$;
    \item Two morphisms $;$ and $||:M\otimes M\to M$ such that $(M,;,e)$ and
      $(M,||,e)$ are monoids;
    \item A 2-cell $\iota$ such that:
% https://q.uiver.app/#q=WzAsNSxbMCwwLCJNXFx0aW1lcyBNIFxcdGltZXMgTVxcdGltZXMgTSJdLFsyLDAsIk1cXHRpbWVzIE1cXHRpbWVzIE1cXHRpbWVzIE0iXSxbMCwyLCJNXFx0aW1lcyBNIl0sWzQsMCwiTVxcdGltZXMgTSJdLFs0LDIsIk0iXSxbMCwxLCJcXGxhbmcgTSwgdCwgTSBcXHJhbmciXSxbMCwyLCJcXGxhbmd8fCx8fFxccmFuZyIsMl0sWzEsMywiXFxsYW5nIDssO1xccmFuZyJdLFszLDQsInx8Il0sWzIsNCwiOyIsMl0sWzIsMywiXFxpb3RhIiwxLHsibGV2ZWwiOjJ9XV0=
\[\begin{tikzcd}
	{M\otimes M \otimes M\otimes M} && {M\otimes M\otimes M\otimes M} && {M\otimes M} \\
	\\
	{M\otimes M} &&&& M
	\arrow["{M\otimes\sigma\otimes M}", from=1-1, to=1-3]
	\arrow["{||\otimes||}"', from=1-1, to=3-1]
	\arrow["{;\otimes;}", from=1-3, to=1-5]
	\arrow["{||}", from=1-5, to=3-5]
	\arrow["\iota"{description}, Rightarrow, from=3-1, to=1-5]
	\arrow["{;}"', from=3-1, to=3-5]
\end{tikzcd}\]
  \end{itemize}.
\end{definition}

\begin{definition}
  A \emph{bialgebra}~\cite{benson1987shuffle} consists of:
  \begin{itemize}
    \item An object $M$ of $\mathcal{C}$;
    \item Two morphisms $\eta:1\to M$ and $\mu:M\otimes M\to M$ such that $(M,\mu,\eta)$ is a monoid;
    \item Two morphisms $\epsilon:M\to 1$ and $\Delta :M\to M\otimes M$ such that
      $(M,\delta,\epsilon)$ is a comonoid;
  \end{itemize}
  such that the following diagrams commute:
% https://q.uiver.app/#q=WzAsNSxbMCwwLCJNXFx0aW1lcyBNIl0sWzIsMCwiTSJdLFswLDIsIk1cXHRpbWVzIE1cXHRpbWVzIE1cXHRpbWVzIE0iXSxbMCw0LCJNXFx0aW1lcyBNXFx0aW1lcyBNXFx0aW1lcyBNIl0sWzIsNCwiTVxcdGltZXMgTSJdLFsxLDQsIlxcRGVsdGEiXSxbMCwxLCJcXG11Il0sWzAsMiwiXFxsYW5nbGVcXERlbHRhLFxcRGVsdGFcXHJhbmdsZSIsMl0sWzIsMywiXFxsYW5nbGUgTSx0LE1cXHJhbmdsZSIsMl0sWzMsNCwiXFxsYW5nbGVcXG11LFxcbXVcXHJhbmdsZSIsMl1d
\[\begin{tikzcd}
	{M\otimes M} && M \\
	\\
	{M\otimes M\otimes M\otimes M} \\
	\\
	{M\otimes M\otimes M\otimes M} && {M\otimes M}
	\arrow["\mu", from=1-1, to=1-3]
	\arrow["{\Delta\otimes\Delta}"', from=1-1, to=3-1]
	\arrow["\Delta", from=1-3, to=5-3]
	\arrow["{M\otimes\sigma\otimes M}"', from=3-1, to=5-1]
	\arrow["{\mu\otimes\mu}"', from=5-1, to=5-3]
\end{tikzcd}\]
% https://q.uiver.app/#q=WzAsNCxbMCwwLCIxIl0sWzIsMCwiTSJdLFsyLDIsIk1cXHRpbWVzIE0iXSxbMCwyLCIxXFx0aW1lcyAxIl0sWzAsMSwiXFxldGEiXSxbMSwyLCJcXERlbHRhIl0sWzMsMiwiXFxldGFcXHRpbWVzXFxldGEiLDJdLFswLDMsIlxcc2ltIiwyLHsibGV2ZWwiOjIsInN0eWxlIjp7ImhlYWQiOnsibmFtZSI6Im5vbmUifX19XV0=
\[\begin{tikzcd}
	1 && M \\
	\\
	{1\otimes 1} && {M\otimes M}
	\arrow["\eta", from=1-1, to=1-3]
	\arrow["\cong"', equals, from=1-1, to=3-1]
	\arrow["\Delta", from=1-3, to=3-3]
	\arrow["{\eta\otimes\eta}"', from=3-1, to=3-3]
\end{tikzcd}
\begin{tikzcd}
	{M\otimes M} && M \\
	\\
	{1\otimes1} && 1
	\arrow["\mu", from=1-1, to=1-3]
	\arrow["{\epsilon\otimes\epsilon}"', from=1-1, to=3-1]
	\arrow["\epsilon", from=1-3, to=3-3]
	\arrow["\cong"', equals, from=3-1, to=3-3]
\end{tikzcd}
% https://q.uiver.app/#q=WzAsMyxbMCwwLCIxIl0sWzIsMiwiMSJdLFsyLDAsIk0iXSxbMiwxLCJcXGVwc2lsb24iXSxbMCwyLCJcXGV0YSJdLFswLDEsIlxcc2ltIiwyLHsibGV2ZWwiOjIsInN0eWxlIjp7ImhlYWQiOnsibmFtZSI6Im5vbmUifX19XV0=
\begin{tikzcd}
	1 && M \\
	\\
	&& 1
	\arrow["\eta", from=1-1, to=1-3]
	\arrow[equals, from=1-1, to=3-3]
	\arrow["\epsilon", from=1-3, to=3-3]
\end{tikzcd}\]
\end{definition}

\begin{example}[The \emph{shuffle bialgebra}~\cite{benson1987shuffle}]
  Let $\Sigma$ be a finite set, then the following relations define a bialgebra
  on $\Sigma^*$ in $\textbf{Rel}$. For all $v,w$ and $z\in\Sigma^*$, we have:
  \begin{itemize}
    \item $1\eta w$ if and only if $w$ is empty;
    \item $w\epsilon 1$ if and only if $w$ is empty;
    \item $(v,w)\mu z$ if and only if $z$ can be obtained by shuffling (while
      preserving relative order) the letters of $v$ and $w$;
    \item $v\Delta (w,z)$ if and only if $v=wz$.
  \end{itemize}
\end{example}

\begin{definition}
  A \emph{dagger category} (or $\dagger$-category) is a category equipped with
  an involutive contravariant functor which acts like the identity on objects.
\end{definition}

\begin{definition}
  A \emph{dagger symmetric monoidal category}~\cite{selinger2007dagger} is a
  category that is both dagger category and a symmetric monoidal category such
  that, for all objects $A,B,C,D$ and all morphisms $f,g$:
  \begin{itemize}
    \item ${(f\otimes g)}^\dagger=f^\dagger\otimes g^\dagger$;
    \item $\alpha_{A,B,C}^\dagger=\alpha_{A,B,C}^{-1}$;
    \item $\lambda_{A}^\dagger=\lambda_{A}^{-1}$;
    \item $\sigma_{A,B}^\dagger=\sigma_{A,B}^{-1}$.
  \end{itemize}
\end{definition}
Let $\textbf{Rel}$ denote the 2-category of sets, relations and
inclusions with its usual dagger symmetric monoidal structure.

For the sake of readability, we will use juxtaposition for the tensor product.
\begin{lemma}\label{comonmon}
  Let $(M, \star, e)$ be a comonoid in a dagger symmetric monoidal category. Then
  $(M, \star^\dagger, e^\dagger)$ is a monoid.
\end{lemma}
\begin{proof}
  Applying the dagger functor to the diagrams proving that $(M, \star, e)$ is a
  comonoid gives diagrams proving that $(M, \star^\dagger, e^\dagger)$ is a monoid.
\end{proof}

%% \begin{lemma}\label{epsdageta}
%%   Let $(M, \mu, \Delta, \eta, \epsilon)$ be a bialgebra in
%%   $\text{\textbf{Rel}}$. Then $\epsilon^\dagger=\eta$.
%% \end{lemma}
%% \begin{proof}
%%   Applying the dagger functor to the last diagram of the bialgebra definition,
%%   we have: $\eta^\dagger\circ\epsilon^\dagger = 1$, then
%%   $\eta=\eta\circ\eta^\dagger\circ\epsilon^\dagger$.
%% \end{proof}

\begin{theorem}
  Let $(M, \mu, \Delta, \eta, \epsilon)$ be a bialgebra in a dagger symmetric
  monoidal category satisfying the two following diagrams and the equation
  $\epsilon^\dagger=\eta$. Then $(M, \mu, \Delta^\dagger, \eta)$ is a concurrent
  monoid.
  % https://q.uiver.app/#q=WzAsMyxbMCwxLCJNXFxvdGltZXMgTSJdLFs0LDEsIk1cXG90aW1lcyBNIl0sWzIsMCwiTSJdLFswLDEsIiIsMCx7ImxldmVsIjoyLCJzdHlsZSI6eyJoZWFkIjp7Im5hbWUiOiJub25lIn19fV0sWzAsMiwiXFxEZWx0YV5cXGRhZ2dlciIsMl0sWzIsMSwiXFxEZWx0YSIsMl0sWzMsMiwiIiwwLHsic2hvcnRlbiI6eyJzb3VyY2UiOjIwfX1dXQ==
\[\begin{tikzcd}
	&& M \\
	{M\otimes M} &&&& {M\otimes M}
	\arrow["\Delta"', from=1-3, to=2-5]
	\arrow["{\Delta^\dagger}"', from=2-1, to=1-3]
	\arrow[""{name=0, anchor=center, inner sep=0}, equals, from=2-1, to=2-5]
	\arrow[between={0.2}{1}, Rightarrow, from=0, to=1-3]
\end{tikzcd}
  \]
% https://q.uiver.app/#q=WzAsMyxbMCwxLCJNIl0sWzQsMSwiTSJdLFsyLDAsIk1cXG90aW1lcyBNIl0sWzAsMSwiIiwwLHsibGV2ZWwiOjIsInN0eWxlIjp7ImhlYWQiOnsibmFtZSI6Im5vbmUifX19XSxbMCwyLCJcXERlbHRhIiwyXSxbMiwxLCJcXERlbHRhXlxcZGFnZ2VyIiwyXSxbMiwzLCIiLDAseyJzaG9ydGVuIjp7InRhcmdldCI6MjB9fV1d
\[\begin{tikzcd}
	&& {M\otimes M} \\
	M &&&& M
	\arrow["{\Delta^\dagger}"', from=1-3, to=2-5]
	\arrow["\Delta"', from=2-1, to=1-3]
	\arrow[""{name=0, anchor=center, inner sep=0}, equals, from=2-1, to=2-5]
	\arrow[between={0}{0.8}, Rightarrow, from=1-3, to=0]
\end{tikzcd}\]
\end{theorem}
In $\textbf{Rel}$, the first diagram states that $\Delta^\dagger$ is entire
(each element of $M\otimes M$ is in relation with at lest one element of $M$),
and the second one states that $\Delta^\dagger$ is \emph{functional} (each
element of $M\otimes M$ is in relation with at most one element of $M$). Thus,
in $\textbf{Rel}$, it suffices that $\Delta^\dagger:M\otimes M\to M$ be a
function.

All these hypotheses are satisfied by the shuffle bialgebra.
\begin{proof}
  By Lemma~\ref{comonmon} and the equation in the hypotheses,
  $(M,\Delta^\dagger, \eta)$ is a monoid. Then, by definition of bialgebra
  $(M,\mu,\eta)$ is a monoid. The only thing left to get a concurrent monoid is
  the 2-cell $\iota$.

  By postcomposing the first diagram of the definition of bialgebra with
  $\Delta^\dagger$ and the second hypothesis of the theorem, the following
  diagram commutes.
% https://q.uiver.app/#q=WzAsNyxbMCwwLCJNTSJdLFsyLDAsIk1NTU0iXSxbNCwwLCJNTU1NIl0sWzYsMCwiTU0iXSxbMiwxLCJNIl0sWzQsMiwiTU0iXSxbNiwzLCJNIl0sWzAsMSwiXFxEZWx0YVxcRGVsdGEiXSxbMSwyLCJNXFxzaWdtYSBNIl0sWzIsMywiXFxtdVxcbXUiXSxbMCw0LCJcXG11Il0sWzQsNSwiXFxEZWx0YSJdLFs1LDYsIlxcRGVsdGFeXFxkYWdnZXIiXSxbMyw2LCJcXERlbHRhXlxcZGFnZ2VyIl0sWzAsNiwiXFxtdSIsMix7ImN1cnZlIjo1fV0sWzExLDE0LCIiLDAseyJzaG9ydGVuIjp7InNvdXJjZSI6MjAsInRhcmdldCI6MjB9fV1d
\[\begin{tikzcd}
	MM && MMMM && MMMM && MM \\
	&& M \\
	&&&& MM \\
	&&&&&& M
	\arrow["{\Delta\Delta}", from=1-1, to=1-3]
	\arrow["\mu", from=1-1, to=2-3]
	\arrow[""{name=0, anchor=center, inner sep=0}, "\mu"', curve={height=30pt}, from=1-1, to=4-7]
	\arrow["{M\sigma M}", from=1-3, to=1-5]
	\arrow["{\mu\mu}", from=1-5, to=1-7]
	\arrow["{\Delta^\dagger}", from=1-7, to=4-7]
	\arrow[""{name=1, anchor=center, inner sep=0}, "\Delta", from=2-3, to=3-5]
	\arrow["{\Delta^\dagger}", from=3-5, to=4-7]
	\arrow[between={0.2}{0.8}, Rightarrow, from=1, to=0]
\end{tikzcd}\]

By precomposing the previous diagram by
$\Delta^\dagger\Delta^\dagger\circ M\sigma M$ and the first
hypothesis, the following diagram commutes.
% https://q.uiver.app/#q=WzAsMTMsWzAsMCwiTU1NTSJdLFszLDAsIk1NTU0iXSxbNiwwLCJNTSJdLFsxLDEsIk1NTU0iXSxbMiwyLCJNTSJdLFszLDMsIk1NTU0iXSxbNCw0LCJNTU1NIl0sWzUsNSwiTU0iXSxbNiw2LCJNIl0sWzEsMywiTU1NTSJdLFsyLDQsIk1NTU0iXSxbMyw1LCJNTSJdLFswLDYsIk1NIl0sWzAsMSwiTVxcc2lnbWEgTSJdLFsxLDIsIlxcRGVsdGFeXFxkYWdnZXJcXERlbHRhXlxcZGFnZ2VyIl0sWzAsMywiTVxcc2lnbWEgTSJdLFszLDQsIlxcRGVsdGFeXFxkYWdnZXJcXERlbHRhXlxcZGFnZ2VyIl0sWzQsNSwiXFxEZWx0YVxcRGVsdGEiXSxbNSw2LCJNXFxzaWdtYSBNIl0sWzYsNywiXFxtdVxcbXUiXSxbNyw4LCJcXERlbHRhXlxcZGFnZ2VyIl0sWzIsOCwiXFxtdSJdLFs1LDIsIiIsMSx7InNob3J0ZW4iOnsic291cmNlIjoxMCwidGFyZ2V0IjoxMH0sImxldmVsIjoyfV0sWzAsOSwiTVxcc2lnbWEgTSIsMl0sWzksMTAsIk1cXHNpZ21hIE0iLDJdLFsxMCwxMSwiXFxtdVxcbXUiLDJdLFsxMSw4LCJcXERlbHRhXlxcZGFnZ2VyIl0sWzAsMTIsIlxcbXVcXG11IiwyXSxbMTIsOCwiXFxEZWx0YV5cXGRhZ2dlciIsMl0sWzEwLDUsIiIsMix7InNob3J0ZW4iOnsic291cmNlIjoxMCwidGFyZ2V0IjoxMH0sImxldmVsIjoyfV1d
\[\begin{tikzcd}[sep=tiny]
	MMMM &&& MMMM &&& MM \\
	& MMMM \\
	&& MM \\
	& MMMM && MMMM \\
	&& MMMM && MMMM \\
	&&& MM && MM \\
	MM &&&&&& M
	\arrow["{M\sigma M}", from=1-1, to=1-4]
	\arrow["{M\sigma M}", from=1-1, to=2-2]
	\arrow["{M\sigma M}"', from=1-1, to=4-2]
	\arrow["{\mu\mu}"', from=1-1, to=7-1]
	\arrow["{\Delta^\dagger\Delta^\dagger}", from=1-4, to=1-7]
	\arrow["\mu", from=1-7, to=7-7]
	\arrow["{\Delta^\dagger\Delta^\dagger}", from=2-2, to=3-3]
	\arrow["{\Delta\Delta}", from=3-3, to=4-4]
	\arrow["{M\sigma M}"', from=4-2, to=5-3]
	\arrow[between={0.1}{0.9}, Rightarrow, from=4-4, to=1-7]
	\arrow["{M\sigma M}", from=4-4, to=5-5]
	\arrow[between={0.1}{0.9}, Rightarrow, from=5-3, to=4-4]
	\arrow["{\mu\mu}"', from=5-3, to=6-4]
	\arrow["{\mu\mu}", from=5-5, to=6-6]
	\arrow["{\Delta^\dagger}", from=6-4, to=7-7]
	\arrow["{\Delta^\dagger}", from=6-6, to=7-7]
	\arrow["{\Delta^\dagger}"', from=7-1, to=7-7]
\end{tikzcd}\]

This diagram gives, by vertical composition, a 2-cell making $(M, \mu, \Delta^\dagger,
\eta)$ a concurrent monoid.
\end{proof}

\begin{theorem}
  TO FIX

  Let $\mathcal{C}$ be a category, and let $(T,\eta,\mu)$ be a monad on
  $\mathcal{C}$. Then, a $\dagger$-category structure on $\mathcal{C}_T$ is
  equivalent to the data of a contravariant functor $T^\dagger$ on $\mathcal{C}$
  and that acts like $T$ on objects that is its own right-adjoint --- i.e.
  $\text{Hom}_{\mathcal{C}}(A,T^\dagger
  B)\equiv\text{Hom}_{\mathcal{C}}(B,T^\dagger A)$.
\end{theorem}
\begin{proof}
  Suppose that $\mathcal{C}_T$ is equipped with a dagger. Define $T^\dagger$ by:
  \[T^\dagger (f:A\to B):=\mu_{TA} \circ T({(\eta_B\circ f)}^\dagger):TB\to TA.\]
  Let us show that $T^\dagger$ is a functor:
  \begin{itemize}
    \item Let $A:\mathcal{C}$. Then $T^\dagger(\text{Id}_A)=\mu_{TA} \circ
      T({(\eta_A)}^\dagger)$, but $\eta_A$ is the identity on $A$ in
      $\mathcal{C}_T$, thus $T^\dagger(\text{Id}_A)=\mu_{TA} \circ
      T\eta_A=\text{Id}_A$ because $T$ is a monad.

    \item Let $f:A\to B$ and $g:B\to C$.
  \end{itemize}

  Conversely, let $T^\dagger$ be a contravariant on $\mathcal{C}$ that acts like
  $T$ on objects, and such that $\text{Hom}_{\mathcal{C}}(A,T^\dagger
  B)\equiv\text{Hom}_{\mathcal{C}}(B,T^\dagger A)$. For any object
  $A:\mathcal{C}$, let $\eta^\ddagger_A:A\to T^2$ be the map obtained by
  applying the previous isomorphism to $\text{Id}_{TA}:TA\to TA$. Define
  $-^\dagger$ by:
  \[{(f:A\to TB)}^\dagger:=T^\dagger f\circ \eta^\ddagger_B.\]
\end{proof}

\section{Profunctors}
We focused earlier on relations. We now study
profunctors~\cite{benabou2000distributors}, a categorical generalisation of
relations.
\begin{definition}
  Let $\mathcal{C}$ be a category. A functor from $\mathcal{C}^{\text{op}}$ to
  $\textbf{Set}$ if called a \emph{presheaf} of $\mathcal{C}$. Presheaves over a
  category form a category $\hat{\mathcal{C}}$ where morphisms are natural
  transformations.
\end{definition}
\begin{definition}
  Let $\mathcal{C}$ and $\mathcal{D}$ be two categories. A \emph{profunctor} ---
  also called \emph{distributor} --- $F$ from $\mathcal{C}$ to $\mathcal{D}$,
  noted $\mathcal{F:\mathcal{C}\rightarrowbar\mathcal{D}}$ is a functor
  $\mathcal{D}^{\text{op}}\times\mathcal{C}\to \textbf{Set}$.
\end{definition}

We cannot express composition of profunctors in terms of Kleisli category for
size reasons: the category of presheaves over a small category is not small
anymore. We thus present necessary concepts to compose profunctors.

\begin{definition}
Let $F:\mathcal{C}\to\mathcal{D}$ and $K:\mathcal{C}\to\mathcal{E}$ be two
functors. A \emph{left Kan extension of $F$ along $K$} consists of a functor
$\text{Lan}_KF:\mathcal{D}\to\mathcal{E}$ equipped with a natural transformation
$\eta:F\Rightarrow\text{Lan}_KF K$ such that every natural transformation
$\theta:F\Rightarrow G K$ with $G:\mathcal{D}\to\mathcal{E}$ factors uniquely
through $\eta$, meaning that there exists a unique natural transformation
$\text{Lan}_KF\Rightarrow G$ satisfying the following equality.
\[
% https://q.uiver.app/#q=WzAsNyxbMCwwLCJcXG1hdGhjYWx7Q30iXSxbNCwwLCJcXG1hdGhjYWx7RX0iXSxbMiwyLCJcXG1hdGhjYWx7RH0iXSxbNiwwLCJcXG1hdGhjYWx7Q30iXSxbMTAsMCwiXFxtYXRoY2Fse0V9Il0sWzgsMiwiXFxtYXRoY2Fse0R9Il0sWzUsMCwiPSJdLFswLDEsIkYiXSxbMCwyLCJLIiwyXSxbMiwxLCJHIiwyXSxbMyw0LCJGIl0sWzMsNSwiSyIsMl0sWzUsNCwiRyIsMSx7ImN1cnZlIjoyfV0sWzUsNCwiXFx0ZXh0e0xhbn1fS0YiLDEseyJjdXJ2ZSI6LTJ9XSxbNywyLCJcXHRoZXRhIiwwLHsic2hvcnRlbiI6eyJzb3VyY2UiOjEwfX1dLFsxMCw1LCJcXGV0YSIsMix7InNob3J0ZW4iOnsic291cmNlIjoxMH19XSxbMTMsMTIsIlxcZXhpc3RzISIsMix7InNob3J0ZW4iOnsic291cmNlIjozMCwidGFyZ2V0IjozMH19XV0=
\begin{tikzcd}
	{\mathcal{C}} &&&& {\mathcal{E}} & {=} & {\mathcal{C}} &&&& {\mathcal{E}} \\
	\\
	&& {\mathcal{D}} &&&&&& {\mathcal{D}}
	\arrow[""{name=0, anchor=center, inner sep=0}, "F", from=1-1, to=1-5]
	\arrow["K"', from=1-1, to=3-3]
	\arrow[""{name=1, anchor=center, inner sep=0}, "F", from=1-7, to=1-11]
	\arrow["K"', from=1-7, to=3-9]
	\arrow["G"', from=3-3, to=1-5]
	\arrow[""{name=2, anchor=center, inner sep=0}, "G"{description}, curve={height=12pt}, from=3-9, to=1-11]
	\arrow[""{name=3, anchor=center, inner sep=0}, "{\text{Lan}_KF}"{description}, curve={height=-12pt}, from=3-9, to=1-11]
	\arrow["\theta", between={0.1}{1}, Rightarrow, from=0, to=3-3]
	\arrow["\eta"', between={0.1}{1}, Rightarrow, from=1, to=3-9]
	\arrow["{\exists!}"', between={0.3}{0.7}, Rightarrow, from=3, to=2]
\end{tikzcd}
\]
\end{definition}

\printbibliography
\end{document}
